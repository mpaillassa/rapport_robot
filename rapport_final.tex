%%%%%%%%%%%%%%%%%%%%%%%%%%%%%%%%%%%%%%%%%%%%%%%%%%%%%%%%%%%%%%%%%%%%%%
% LaTeX Example: Project Report
%
% Source: http://www.howtotex.com
%
% Feel free to distribute this example, but please keep the referral
% to howtotex.com
% Date: March 2011 
% 
%%%%%%%%%%%%%%%%%%%%%%%%%%%%%%%%%%%%%%%%%%%%%%%%%%%%%%%%%%%%%%%%%%%%%%

% Edit the title below to update the display in My Documents
%\title{Project Report}
%
%%% Preamble
\documentclass[paper=a4,12pt]{article}
\usepackage[T1]{fontenc}
\usepackage{kpfonts}
%\usepackage[fontsize=12pt]{scrextend}
\usepackage[utf8]{inputenc}
\usepackage[french]{babel}

% English language/hyphenation

\usepackage{amsmath,amsfonts,amsthm} % Math packages
%\usepackage[pdftex]{graphicx}	
\usepackage{url}
\usepackage[bottom=10em]{geometry}
\usepackage{float}
\usepackage{xcolor}
\usepackage{enumitem}
\usepackage{rotating}
\usepackage{lscape}
 


%%% Custom sectioning
%\usepackage{sectsty}
%\allsectionsfont{\normalfont\scshape}

%% Language definition package (for XML Annexe)
\usepackage{listings}
\usepackage{color}

%% Local modification of margins
%\newenvironment{changemargin}[2]{\begin{list}{}{%
%      \setlength{\topsep}{0pt}%
%      \setlength{\leftmargin}{0pt}%
%      \setlength{\rightmargin}{0pt}%
%      \setlength{\listparindent}{\parindent}%
%      \setlength{\itemindent}{\parindent}%
%      \setlength{\parsep}{0pt plus 1pt}%
%      \addtolength{\leftmargin}{#1}%
%      \addtolength{\rightmargin}{#2}%
%    }\item }{\end{list}}
%%%

%%% Custom headers/footers (fancyhdr package)
%\usepackage{fancyhdr}
%\pagestyle{fancyplain}
%\fancyhead{}											% No page header
%\fancyfoot[L]{}											% Empty 
%\fancyfoot[C]{}											% Empty
%\fancyfoot[C]{\thepage}									% Pagenumbering
%\renewcommand{\headrulewidth}{0pt}			% Remove header underlines
%\renewcommand{\footrulewidth}{0pt}				% Remove footer underlines
%\setlength{\headheight}{13.6pt}


%%% Equation and float numbering
\numberwithin{equation}{section}		% Equationnumbering: section.eq#
\numberwithin{figure}{section}			% Figurenumbering: section.fig#
\numberwithin{table}{section}				% Tablenumbering: section.tab#

%Graphics path
%\graphicspath{./Images/}

%%% Maketitle metadata
\newcommand{\horrule}[1]{\rule{\linewidth}{#1}} 	% Horizontal rule

\title{
  %\vspace{-1in} 			
  %\usefont{OT1}{bch}{b}{n}
  \horrule{1.5pt} \\[0.5cm]	
  \Huge \textbf{Projet Rain Of Music : \\ Rapport} \\ [20pt]
 % \huge TM, \\ Université d'Osaka \\ [15pt]
 	%	\vspace{1cm}  
    \LARGE Année scolaire 2015-2016 \\ 
  \horrule{1.5pt} \\[0.5cm]
  %
}

\author{						
    \LARGE \underline{Encadrants} : Jean-Michaël Celerier, \\
   					\LARGE	\hspace{5cm} Myriam De Sainte-Catherine\\			
   	\vspace{1cm} 
   	\normalfont
   	\LARGE Élèves :  Akané LEVY, Maxime PAILLASSA    
}
\date{}

%%% Begin document
\begin{document}
\graphicspath{{./imgs/}{.}}
\maketitle

\begin{figure}[H]
  \centering\includegraphics[scale=1.2]{logo_enseirb.png}
\end{figure}

\newpage

\tableofcontents

\newpage
\normalsize


% contexte du projet
\section*{Introduction}

Le but du projet Rain of music est de pouvoir scénariser un spectacle impliquant des robots et de la musique. Deux types de robots sont envisagés: des métabots (robots terrestres) et des drones (robots volant sans pilote). Ces robots doivent réaliser une chorégraphie dans un espace donné et embarquer des haut parleurs pour émettre des sons. Ce projet pluridisplinaire implique trois parties: des étudiants en arts pour la conception et l'écriture de la chorégraphie, des étudiants roboticiens pour gérer les problématiques de communication et de localisation des robots, et des étudiants en technologies multimédia (nous-mêmes) pour les sons et l'interface qui permettra aux artistes d'écrire et de simuler des chorégraphies. \\

Notre travail s'inscrit dans le début du projet (3 premiers mois) qui durera bien plus longtemps pour arriver aux objectifs finaux cités précédemment. Dans ce contexte, nous avons dû adapter nos objectifs. Ainsi, nous nous sommes fixés de réaliser un logiciel qui permettra de simuler une chorégraphie dans un espace 3D pour aider les artistes dans l'écriture d'une chorégraphie. Dans ce rapport, nous allons d'abord faire un état de l'art. Ensuite, nous allons présenter les outils que l'on a choisi pour remplir nos objectifs. Puis, nous nous intéresserons à l'architecture du logiciel et son implémentation. Enfin, nous évaluerons le logiciel en testant plusieurs scénarios différents. \\

Avant de continuer, précisons un peu les objectifs et les deux principales problématiques liées au logiciel.
Comme dit précédemment, notre travail consistera à écrire un logiciel permettant de visualiser et de simuler une chorégraphie écrite par les artistes. Cette chorégraphie est écrite grâce au logiciel i-score, qui est un séquenceur: il permet d'écrire des scénarios en programmant des évènements dans le temps. Notre travail impliquera alors deux problématiques: 
\begin{itemize}
\item la récupération des données d'i-score dans le logiciel, de manière à connaître les positions des robots.
\item l'affichage des robots dans une scène 3D, qui impliquera d'autres problématiques comme la gestion de collisions.
\end{itemize} 

\newpage

% objectifs et les motivations du projet
\input{objectifs.tex}

% ce qui existe
\section{Etat de l'art}

Dans cette partie, nous allons présenter les moyens existant nous permettant de répondre à nos objectifs. Nous verrons d'abord les moyens pour communiquer avec i-score. Ensuite, nous verrons ce qui a été fait en terme de visualisation et de simulation de performances scéniques impliquant des robots. Enfin, des bibliothèques graphiques 3D seront présentées.


\subsection{i-score}

Tout le travail réalisé devra s'interfacer avec le logiciel i-score utilisé pour écrire les chorégraphies. Pour communiquer avec un autre logiciel, i-score intègre le format OSC qui utilise le protocole UDP. Un message OSC se compose comme ceci:
\begin{lstlisting}
<address pattern> <data type> <data>
\end{lstlisting}
Address pattern représente l'URL désignant la destination du message, et data type une chaîne de caractères indiquant le type de données contenues dans data.
Cependant, le protocole Minuit peut aussi être utilisé : ce protocole s'ajoute au protocole OSC, et permet de mieux organiser les données transmises. Les adresses OSC sont organisées dans un arbre et le logiciel pourra demander via le protocole Minuit à i-score de lui communiquer la valeur de tel ou tel paramètre. Nous verrons dans la suite de ce rapport une API qui permet de mettre en oeuvre ces fonctions pour communiquer avec i-score.

\subsection{Visualisation et simulation}

Cette partie présentera quelques articles scientifiques se rapprochant de notre projet autour des thématiques telles que la visualisation de scène 3D d'une peformance scénique et la modélisation de robot dans un espace 3D. 

Dans le domaine de visualisation de performance scénique 3D, un des objectifs est d'obtenir un rendu se rapprochant le plus possible de la réalité de façon à ce que l'on puisse avoir un aperçu réaliste de la performance conçue virtuellement. C'est dans cette optique que le logiciel de visualisation et simulation 3D StageViz\cite{StageViz} a été développé . Ce logiciel permet de visualiser en 3D en temps réel une performance scénique qui a été conçue virtuellement à l'aide d'un autre logiciel dédié à cela. Dans la forme, nous retrouvons ici, un cas très similaire à notre objectif avec la présence de visualisation mais aussi d'interopérabilité avec un outil de conception scénique.

StageViz se base sur trois éléments principaux que l'on doit aussi retrouver dans notre projet : les objets 3D, la manipulation de la scène et la chronologie. Dans le cadre de notre projet, les objets 3D représenteront les robots et la scène pourra être visionner sous tous les angles grâce à une camera mobile. En ce qui concerne la composante temporelle, elle sera gérée au niveau d'i-score qui nous envoie les données sur les robots au fur et à mesure. 

Une grande différence entre StageViz et notre projet réside dans l'utilisation du rendu 3D. StageViz serait un outil pour une simulation réaliste pour visionner les différents états des objets dans la scène au cours de la performance. Notre but serait de visualiser l'évolution des positions des objets dans la scène pour aider l'écriture de la chorégraphie. L'idée est donc d'avoir un rendu permettant une vision simplifiée et globale de la chorégraphie avec tous les robots.

Dans le domaine de performance par les robots, une équipe de recherche a étudié la modélisation des différents mouvements d'un robot dans le but de visualiser une performance \cite{robotArt}. Le robot n'est pas considéré en tant qu'un seul objet avec une position mais comme un ensemble de composants possédant des paramètres (degré de liberté) et organisés hiérarchiquement. Ainsi lorsqu'un mouvement est exécuté au niveau d'une composante, il est appliqué à toutes ses composantes filles.



\subsection{Bibliothèques graphiques}
Dans cette partie, nous allons présenter différentes bibliothèques graphiques 3D que nous pourrions utiliser: OpenSceneGraph, OpenFrameworks, Ogre, Blender, Unity, Qt3D, Babylon.js et Three.js. Toutes ces bibliothèques sont basées sur OpenGL (ou WebGL). Nous n'avons pas voulu utiliser uniquement OpenGL car cela aurait demandé plus de code et de temps pour un résultat équivalent. 

Elles sont déployables sur pratiquement toutes les plateformes (y compris smartphones et tablettes). Celles qui sont orientées objets permettent d'avoir une programmation plus haut niveau tout en profitant des performances d'OpenGL. Elles intègrent également des packages additionnels pour faire des rendus 3D plus poussés. 

Babylon.js et Three.js se déploient sur navigateur via javascript et utilisent WebGL.


\subsubsection{openFrameworks}

openFrameworks est open source, sous license MIT (compatible GPL), basée sur du C++ et OpenGL. Elle permet également d'intégrer et d'utiliser d'autres bibliothèques comme OpenCV par exemple, qui pourrait être utilisée pour de la gestion de trajectoires et de collisions, et de gérer des flux audio, ce qui peut s'avérer utile dans l'optique d'embarquer des haut parleurs sur les robots pour générer des sons. 

De plus, une des lignes directrices de ce projet actif est la simplicité: la bibliothèque est faite de manière à pouvoir être utilisée avec un minimum de connaissances, et propose des tutoriaux sur des bases comme l'OpenGL ou la programmation orientée objet. Ceci est un atout dans notre contexte puisque nous avons peu de temps pour implémenter le projet qui sera ensuite plus facile à reprendre par d'autres personnes.

\subsubsection{OpenSceneGraph}

OpenSceneGraph est une bibliothèque 3D open source distribuée sous la license OSGPL (OpenSceneGraph Public License) basée sur la license LGPL. Elle est écrite en C++ et en OpenGL et est utilisée dans de nombreux domaines: réalité virtuelle, jeux vidéos, visualisation scientifique et simulation. 

Le monde 3D virtuel est représenté par un graphe dont les noeuds sont logiquement et spatialement organisés. Cette vision pourrait être un problème dans le cadre du projet car nous avons peu de temps et le projet est destiné à être repris par d'autres personnes. De plus, la communauté OpenSceneGraph ne semble pas très active, ce qui pourrait indiquer que peu de gens l'utilise, et probablement les futures personnes reprenant le projet. 

\subsubsection{Ogre}

Ogre est encore une biliothèque open source, sous license MIT, écrite en C++ et OpenGL. Elle est beaucoup utilisée pour réaliser des jeux vidéos et faire des modélisations 3D de personnages. Dans notre cas, nous n'avons pas besoin d'aller si loin au niveau du rendu 3D. L'objectif est plus d'avoir un rendu clair et fidèle à la chorégraphie.  


\subsubsection{Qt3D}

Qt est un framework utilisé pour faire des interfaces graphiques. Qt est open source, écrit en C++ sous license GNU GPL ou LGPL selon les versions. Le module Qt3D permet de faire de la modélisation 3D. Cependant, ce module est encore en cours de développement. L'utiliser serait donc prendre le risque de devoir changer plus tard du code déjà écrit, ce qui peut être problématique surtout après que le projet soit repris par d'autres personnes. 

\subsubsection{Blender et Unity}

Blender et Unity sont deux logiciels de modélisation et d'animation 3D écrits en C, C++ et Python (Blender) et C\# (Unity). Blender est open source alors que Unity ne l'est pas, même si des licenses gratuites sont proposées. Un des avantages de ces logiciels est leur communauté qui est très active.

De la même manière qu'avec Ogre, ces logiciels se dirigent plus vers du rendu 3D pour des jeux vidéos et de la modélisation 3D alors que notre objectif premier est la visualisation. 

De plus, ces logiciels proposent des interfaces pour créer facilement des objets 3D ou des textures par exemple mais ce qui nous intéresse plus est la gestion de ces objets, et cela nous parait mieux de pouvoir la gérer avec du code plutôt que dans une interface graphique. 
				

\subsubsection{Babylon.js et Three.js}

Babylon.js et Three.js sont des bibliothèques javascript qui utilisent WebGL pour avoir un rendu visuel dans le navigateur. Les deux sont open source, mais Three.js est sous license MIT alors que Babylon.js est sous une license Apache, qui cherche encore à être compatible avec la license GPL. L'avantage de ces bibliothèques est que seul un navigateur est nécessaire pour visualiser la scène, ce qui les rend très accessible. Cependant, plus le projet avancera et se complexifiera (possibilité de modifier la chorégraphie par exemple), plus il nous semble compliqué d'utiliser javascript.








% choix d'outils
  % openframeworks
  % OSSIA API / Jamoma
\section{Outils utilisés}

Dans cette partie, nous allons présenter les outils que l'on a utilisé pour réaliser simulationRainOfMusic. Nous présenterons d'abord l'API d'OSSIA qui a été utilisée pour la communication avec i-score. Ensuite, nous représenterons les bibliothèques graphiques 3D évoquées précédemment et expliquerons le choix que nous avons fait.

\subsection{L'API d'OSSIA}

OSSIA est l'acronyme pour Open Scenario System for Interactive Application. Comme son nom l'indique, c'est un projet de développement d'outils logiciels de scénarisation pour des applications intéractives. Le logiciel i-score, qui est un séquenceur qui sert à écrire la chorégraphie, a été écrit dans le cadre de ce projet. 

Dans simulationRainOfMusic, nous utilisons l'API implémentée dans le projet OSSIA pour pouvoir communiquer avec i-score via le protocole minuit. L'API nous permet de réaliser les principales fonctions dont nous avons besoin: publier des données vers i-score et récupérer des données d'i-score depuis simulationRainOfMusic.  Minuit fonctionne en organisant les données sous forme d'arbre. Dans simulationRainOfMusic, on va créer des robots qui ont certains attributs et l'API va nous permettre de créer les noeuds de l'arbre correspondant aux robots et à leurs attributs et de les envoyer à i-score. L'API va également nous permettre d'écouter les attributs des robots pour les actualiser dans simulationRainOfMusic et permettre l'affichage des robots. 

Des exemples simples de fonctionnement sont disponibles dans les exemples Minuit_publication et Minuit_Exploration de la documentation de l'API.

\subsection{Bibliothèques graphiques 3D}

Dans cette partie, un bilan de l'étude des différents biblitohèques sera présentée sous forme de tableau avec cinq critères jugés importants dans le choix de la biliothèque. Parmi ces cinq, ne seront pas présentées ceux élémentaires telles que la présence d'un moteur 3D complet, l'aspect multi-plateforme, pour lesquelles il semble évident que toutes les bibliothèques citées jusque là les remplissent.

Un des critères qui nous a paru essentiel dans le développement du logiciel est l'auto-suffisance de la bibliothèque. C'est-à-dire un choix de possibilité assez large proposé dans la bibliothèque de base pour éviter l'utilisation de plusieurs outils annexes. Par exemple, nous avons vu pour Babylon.js qu'il faudrait l'utilisation d'autres outils (Node.js ?) pour pouvoir réaliser la réception de messages UDP. 

Un critère moins important dans une première partie du développement mais qui serait nécessaire par la suite est la présence d'un moteur physique. En effet, il serait intéressant d'avoir des fonctionnalités pour la détection de collisions, permettant ainsi d'alerter l'artiste lors de la conception de la chorégraphie.

Plus généralement, certains critères choisis pour ce bilan ont été réfléchi en vue de l'interopérabilité de notre logiciel avec i-score : l'aspect open source (notamment license GPL) et le langage utilsé. En effet, il nous semblait plus pertinant de respecter une unicité de langage avec i-score, et donc choisir le C++. 
D'autres critères ont été pensé dans l'optique de la maintenabilité, notamment la facilité à prendre en main les différentes fonctionnalités de la bibliothèque, la stabilité des versions proposées, la présence d'une communauté active (tant au niveau développeurs qu'utilisateurs).



Les critères qui ont donc été choisi pour ce bilan sont : 
\begin{itemize}
\item open source et notamment license GPL comme le logiciel i-score 
\item détection de collisions (moteur physique)
\item auto-suffisant (facilité d'utilisation d'UDP)
\item unicité du langage avec i-score (C++)
\item facilité à prendre en main
\item maintenabilité (bonne communauté active)
\end{itemize} 
 

\newpage
\begin{landscape}
\hspace{-4.5cm} 
\begin{tabular}{l|c|c|c|c|c|c|c|c}
Bibliothèque & open source & moteur physique & auto-suffisant & langage & prise en main & maintenabilité & stabilité & communauté active\\
\hline
openFrameworks & Oui & Oui & Oui & C++ & Bonne & Oui & Oui & Fort\\
OpenSceneGraph & Oui & Oui & Oui & C++ & Moyenne & Oui & Oui & Moyen\\ 
OGRE & Oui & Oui & Oui & C++ & Moyenne & Oui & Oui & Fort\\
Qt3D & Oui & Oui & Oui & C++ & Moyenne & Oui & Non & Faible\\
Unity & Non & Oui & Oui & C\# & Bonne & Oui & Oui & Fort\\
Babylon.js  & Non & Oui & Oui & JavaScript & Bonne & Oui & Oui & Moyen/Fort
\end{tabular}
\end{landscape}

Finalement, presque toutes les bibliothèques conviennent en terme de performances et de fonctionnalités. Le choix s'est donc porté sur les derniers critères (prise en main, maintenabilité, stabilité et communauté) qui s'avèrent être aussi importants dans le cadre de notre projet. Ainsi, nous avons choisi d'utiliser openFrameworks, qui semble être la meilleure par rapport à l'ensemble des critères.  


% architecture et implémentation
  % archi
  % implementation
    % visualisation
    % communication
    % simulation
\section{Réalisation du logiciel}

Maintenant que nous avons vu les outils utilisés pour réaliser simulationRainOfMusic, nous allons présenter son architecture et les éléments importants de son implémentation.

\subsection{Architecture}
Dans les premières étapes de la réalisation du logiciel, nous avons établi l'architecture en respectant certains points qui nous paraissaient essentiels à ntore projet.

Tout d'abord, notre logiciel étant un logiciel de visualisation avant tout, il nous semblait plus clair de choisir une architecture modèle-vue. Cela nous permettait d'avoir une séparation nette entre la partie données des robots et la partie modélisation et affichage 3D. 

Ensuite, sachant que le projet contient des robots Metabot mais aussi drones, il nous avait été demandé à ce que notre logiciel soit modulaire, facilitant ainsi l'ajout de nouvaux types de robots. C'est pour cela que nous avons choisi dans notre implémentation d'utiliser le mécanisme en C++ des \verb|template| pour assurer ce côté modulaire.

Enfin, une attention particulière a été portée à la classe \verb|Parameter|, jouant le rôle d'intermédiaire entre i-score et l'interface utilisateur de notre logiciel. Cette classe occupe une place importante dans le projet puisqu'elle permet de mettre à jour les valeurs en fonction des modifications apportées à notre interface et à i-score.

Par ailleurs, l'intégration de la classe \verb|Parameter| dans l'architecture s'est déroulée naturellement : elle prend le rôle du contrôleur qui met à jour les valeurs du modèle en fonction des données modifiées sur la vue qui est aussi l'interface utilisateur.

		
\subsection{Implémentation}

Dans cette partie, nous allons présenter simulationRainOfMusic en s'intéressant d'un peu plus près au code qui permettent de remplir les trois objectifs principaux: la visualisation, la communication et la simulation.

\subsubsection{Visualisation}

La mise en place de la scène 3D se fait assez facilement car la caméra est déjà implémentée dans openFrameworks avec la classe (\verb|EasyCam|). L'utilisateur a la possibilité de se déplacer et de zoomer dans la scène 3D grâce à la souris. Les robots sont représentés par des modèles 3D.

L'utilisateur peut également sélectionner un robot. Lorsq'un robot est sélectionné, différentes caractéristiques le concernant s'affichent en haut à gauche de la fenêtre de simulation. Ces caractéristiques sont son identifiant, son type (métabot ou drone) et sa position. La sélection se fait en détectant la couleur du robot.

Si deux robots risquent de rentrer en collision, ceux-ci sont entourés d'un cercle rouge afin que ce soit bien visible.

En plus de ces informations, un onglet permet de visualiser et de modifier les positions et vitesses des robots via des sliders. Un emplacement dans l'onglet affiche également si les robots sont en collision ou pas. Une fois des vitesses choisies, un bouton play permet de voir la chorégraphie.


possibilité de rajouter des axes x y z en plus pour mieux situer le robot dans l'espace

\subsubsection{Communication}

La communication avec i-score est possible grâce à l'API d'OSSIA et à la classe \verb|Network|. Cette dernière permet d'initier la communication avec i-score. A la construction d'un objet Network, on lance l'exécution d'une fonction sur un thread qui fait la publication: elle déclare le programme comme un objet réseau, que l'on peut ajouter dans i-score en lui donnant son nom et le numéro de port associé, et qui constitue la racine de l'arbre du réseau. Elle envoie ensuite sur le réseau le noeud 'scène' en le déclarant comme noeud fils du programme.

A l'instanciation d'un robot, celui-ci est ajouté à l'arbre du réseau comme fils du noeud 'scène'. Chacun de ses attributs est aussi publié. Lorsqu'un attribut dans l'arbre publié sur le réseau est modifié via l'interface utilisateur d'i-score, sa valeur est mise à jour dans simulationRainOfMusic grâce à une fonction (\verb|Callback|).

Pour ajouter la communication entre simulationRainOfMusic et son interface, la classe \verb|Parameter| a été créee. Elle encapsule la classe \verb|ofParameter| de openFrameworks. La classe \verb|ofParameter| permet de lier un listener vers l'interface à une valeur. Le listener est une fonctionnalité qui permet d'être constamment à l'écoute de toute modification opérée au niveau de l'interface. Par exemple, un ofParameter peut être utilisé pour lier une valeur position en x avec un slider dans l'interface utilisateur : le fait de bouger le slider va mettre à jour la valeur représentant la position en x. De cette façon, utiliser une variable de type \verb|ofParameter| permet de lier un attribut de robot à l'interface utilisateur. 

Ainsi, la classe \verb|Parameter| encapsulant \verb|ofParameter| utilise un listener \verb|Callback| vers i-score et un listener vers l'interface de simulationRainOfMusic.


\subsubsection{Simulation}

détection de collision manuelle : bibliothèques graphiques ne compilaient pas lors de l'implémentation de la collision (Bullet)...



% évaluation et validation
\section{Évaluation du projet}
\subsection{Validation du logiciel} %validation des fonctionnalités du logiciel ..?
\subsubsection{Visualisation}
parler de fps, de maniabilité de la scène 3D, de la GUI

\subsubsection{Communication réseau}
parler de latence, de l'établissement de la connection (qui foire de temps en temps, essayer de faire peut être plusiuers tests pour établir une moyenne), de la perte de paquets éventuelle ..?

\subsubsection{Simulation}
parler des aspects simulation, la proximité à la réalité (comparer le modèle théorique au modèle réelle si possible, mais les tests réelles avec les metabots risquent d'être compromis ...)

\subsubsection{Bilan validation}
reprendre sous forme de tableau ce qui a été cité précédemment ?

\subsection{Gestion du projet}
\subsubsection{Calendrier}
Comparaison du calendrier initial avec celui réel

\subsubsection{Difficultés rencontrées}


% conclusion
\section{Conclusion}
\subsection{Bilan}


\subsection{Apport de notre travail dans le projet RainOfMusic}
ce que notre travail a finalement apporté au projet dans globalité (possibilité de simuler mais aussi d'écrie la chorégraphie avec notre logiciel : un plus qui n'avait pas vraiment explicité dans les besoins au début)

\subsection{Améliorations envisageables}
initialisation de la scène (ajuster les dimensions et les échelles) dans l'interface utilisateur et non dans le code 

rajout de façon dynamique des robots via l'interface utilisateur (et non en dur comme dans l'implémentation actuelle)

côté simulation, mouvements des drones et pannes probables ..?


%\subsection{Apport personnel}
%peut être écrire un paragraphe chacun avec ce que ça nous apportait ..?

  \newpage

  \bibliographystyle{plain}
  \bibliography{bibli.bib}
  \newpage

  \appendix
  \part*{Annexes}
  %\section{Manuel d'utilisation}
%manuel d'utilisation pour artiste (léger, avec i-score create device) et pour développeur () 

\section{Chorégraphie}
Voici la chorégraphie sous forme de storyboard conçue par les étudiants d'Art de l'université de Bilbao. On peut distinguer que le spectacle est divisé en plusieurs parties, l'image du storyboard correspondant est indiqué entre parenthèses : le réveil (1), la rencontre (2), la danse (3-10), la percussion avec un rythme particulier engendré par les pattes des robots (11-12), la dernière ronde (13), le retrait (14-18).

\hspace*{-2cm}
\includepdf[pages={1,2},scale=0.9]{storyboard}


%%% End document
\end{document}
