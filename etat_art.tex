\section{Etat de l'art}

Dans cette partie, nous allons présenter les moyens existant nous permettant de répondre à nos objectifs. Nous verrons d'abord les moyens pour communiquer avec i-score. Ensuite, nous verrons ce qui a été fait en terme de visualisation et de simulation de performances scéniques impliquant des robots. Enfin, des bibliothèques graphiques 3D seront présentées.


\subsection{i-score}

Tout le travail réalisé devra s'interfacer avec le logiciel i-score utilisé pour écrire les chorégraphies. Pour communiquer avec un autre logiciel, i-score intègre le format OSC qui utilise le protocole UDP. Un message OSC se compose comme ceci:
\begin{lstlisting}
<address pattern> <data type> <data>
\end{lstlisting}
Address pattern représente l'URL désignant la destination du message, et data type une chaîne de caractères indiquant le type de données contenues dans data.
Cependant, le protocole Minuit peut aussi être utilisé : ce protocole s'ajoute au protocole OSC, et permet de mieux organiser les données transmises. Les adresses OSC sont organisées dans un arbre et le logiciel pourra demander via le protocole Minuit à i-score de lui communiquer la valeur de tel ou tel paramètre. Nous verrons dans la suite de ce rapport une API qui permet de mettre en oeuvre ces fonctions pour communiquer avec i-score.

\subsection{Visualisation et simulation}

Cette partie présentera quelques articles scientifiques se rapprochant de notre projet autour des thématiques telles que la visualisation de scène 3D d'une peformance scénique et la modélisation de robot dans un espace 3D. 

Dans le domaine de visualisation de performance scénique 3D, un des objectifs est d'obtenir un rendu se rapprochant le plus possible de la réalité de façon à ce que l'on puisse avoir un aperçu réaliste de la performance conçue virtuellement. C'est dans cette optique que le logiciel de visualisation et simulation 3D StageViz\cite{StageViz} a été développé . Ce logiciel permet de visualiser en 3D en temps réel une performance scénique qui a été conçue virtuellement à l'aide d'un autre logiciel dédié à cela. Dans la forme, nous retrouvons ici, un cas très similaire à notre objectif avec la présence de visualisation mais aussi d'interopérabilité avec un outil de conception scénique.

StageViz se base sur trois éléments principaux que l'on doit aussi retrouver dans notre projet : les objets 3D, la manipulation de la scène et la chronologie. Dans le cadre de notre projet, les objets 3D représenteront les robots et la scène pourra être visionner sous tous les angles grâce à une camera mobile. En ce qui concerne la composante temporelle, elle sera gérée au niveau d'i-score qui nous envoie les données sur les robots au fur et à mesure. 

Une grande différence entre StageViz et notre projet réside dans l'utilisation du rendu 3D. StageViz serait un outil pour une simulation réaliste pour visionner les différents états des objets dans la scène au cours de la performance. Notre but serait de visualiser l'évolution des positions des objets dans la scène pour aider l'écriture de la chorégraphie. L'idée est donc d'avoir un rendu permettant une vision simplifiée et globale de la chorégraphie avec tous les robots.

Dans le domaine de performance par les robots, une équipe de recherche a étudié la modélisation des différents mouvements d'un robot dans le but de visualiser une performance \cite{robotArt}. Le robot n'est pas considéré en tant qu'un seul objet avec une position mais comme un ensemble de composants possédant des paramètres (degré de liberté) et organisés hiérarchiquement. Ainsi lorsqu'un mouvement est exécuté au niveau d'une composante, il est appliqué à toutes ses composantes filles.



\subsection{Bibliothèques graphiques}
Dans cette partie, nous allons présenter différentes bibliothèques graphiques 3D que nous pourrions utiliser: OpenSceneGraph, OpenFrameworks, Ogre, Blender, Unity, Qt3D, Babylon.js et Three.js. Toutes ces bibliothèques sont basées sur OpenGL (ou WebGL). Nous n'avons pas voulu utiliser uniquement OpenGL car cela aurait demandé plus de code et de temps pour un résultat équivalent. 

Elles sont déployables sur pratiquement toutes les plateformes (y compris smartphones et tablettes). Celles qui sont orientées objets permettent d'avoir une programmation plus haut niveau tout en profitant des performances d'OpenGL. Elles intègrent également des packages additionnels pour faire des rendus 3D plus poussés. 

Babylon.js et Three.js se déploient sur navigateur via javascript et utilisent WebGL.


\subsubsection{openFrameworks}

openFrameworks est open source, sous license MIT (compatible GPL), basée sur du C++ et OpenGL. Elle permet également d'intégrer et d'utiliser d'autres bibliothèques comme OpenCV par exemple, qui pourrait être utilisée pour de la gestion de trajectoires et de collisions, et de gérer des flux audio, ce qui peut s'avérer utile dans l'optique d'embarquer des haut parleurs sur les robots pour générer des sons. 

De plus, une des lignes directrices de ce projet actif est la simplicité: la bibliothèque est faite de manière à pouvoir être utilisée avec un minimum de connaissances, et propose des tutoriaux sur des bases comme l'OpenGL ou la programmation orientée objet. Ceci est un atout dans notre contexte puisque nous avons peu de temps pour implémenter le projet qui sera ensuite plus facile à reprendre par d'autres personnes.

\subsubsection{OpenSceneGraph}

OpenSceneGraph est une bibliothèque 3D open source distribuée sous la license OSGPL (OpenSceneGraph Public License) basée sur la license LGPL. Elle est écrite en C++ et en OpenGL et est utilisée dans de nombreux domaines: réalité virtuelle, jeux vidéos, visualisation scientifique et simulation. 

Le monde 3D virtuel est représenté par un graphe dont les noeuds sont logiquement et spatialement organisés. Cette vision pourrait être un problème dans le cadre du projet car nous avons peu de temps et le projet est destiné à être repris par d'autres personnes. De plus, la communauté OpenSceneGraph ne semble pas très active, ce qui pourrait indiquer que peu de gens l'utilise, et probablement les futures personnes reprenant le projet. 

\subsubsection{Ogre}

Ogre est encore une biliothèque open source, sous license MIT, écrite en C++ et OpenGL. Elle est beaucoup utilisée pour réaliser des jeux vidéos et faire des modélisations 3D de personnages. Dans notre cas, nous n'avons pas besoin d'aller si loin au niveau du rendu 3D. L'objectif est plus d'avoir un rendu clair et fidèle à la chorégraphie.  


\subsubsection{Qt3D}

Qt est un framework utilisé pour faire des interfaces graphiques. Qt est open source, écrit en C++ sous license GNU GPL ou LGPL selon les versions. Le module Qt3D permet de faire de la modélisation 3D. Cependant, ce module est encore en cours de développement. L'utiliser serait donc prendre le risque de devoir changer plus tard du code déjà écrit, ce qui peut être problématique surtout après que le projet soit repris par d'autres personnes. 

\subsubsection{Blender et Unity}

Blender et Unity sont deux logiciels de modélisation et d'animation 3D écrits en C, C++ et Python (Blender) et C\# (Unity). Blender est open source alors que Unity ne l'est pas, même si des licenses gratuites sont proposées. Un des avantages de ces logiciels est leur communauté qui est très active.

De la même manière qu'avec Ogre, ces logiciels se dirigent plus vers du rendu 3D pour des jeux vidéos et de la modélisation 3D alors que notre objectif premier est la visualisation. 

De plus, ces logiciels proposent des interfaces pour créer facilement des objets 3D ou des textures par exemple mais ce qui nous intéresse plus est la gestion de ces objets, et cela nous parait mieux de pouvoir la gérer avec du code plutôt que dans une interface graphique. 
				

\subsubsection{Babylon.js et Three.js}

Babylon.js et Three.js sont des bibliothèques javascript qui utilisent WebGL pour avoir un rendu visuel dans le navigateur. Les deux sont open source, mais Three.js est sous license MIT alors que Babylon.js est sous une license Apache, qui cherche encore à être compatible avec la license GPL. L'avantage de ces bibliothèques est que seul un navigateur est nécessaire pour visualiser la scène, ce qui les rend très accessible. Cependant, plus le projet avancera et se complexifiera (possibilité de modifier la chorégraphie par exemple), plus il nous semble compliqué d'utiliser javascript.






