\section{Évaluation du projet}
\subsection{Validation du logiciel} %validation des fonctionnalités du logiciel ..?
\subsubsection{Visualisation}
parler de fps, de maniabilité de la scène 3D, de la GUI

\subsubsection{Communication réseau}
parler de latence, de l'établissement de la connection (qui foire de temps en temps, essayer de faire peut être plusiuers tests pour établir une moyenne), de la perte de paquets éventuelle ..?

\subsubsection{Simulation}
parler des aspects simulation, la proximité à la réalité (comparer le modèle théorique au modèle réelle si possible, mais les tests réelles avec les metabots risquent d'être compromis ...)

\subsubsection{Bilan validation}
reprendre sous forme de tableau ce qui a été cité précédemment ?

\subsection{Gestion du projet: calendrier et difficultés}

Dans cette partie, nous verrons la gestion du projet par rapport au calendrier que nous nous étions fixé. Puis nous verrons les difficultés que nous avons rencontrées. Un calendrier bilan est représenté sur la figure suivante \ref{cal}:

\begin{figure}[H]
  \begin{center}
  	\includegraphics[scale=0.7]{imgs/calendrierbis.png}
  	\caption{Calendrier bilan. En noir ce qui était prévu. En rouge ce qui n'a pas été fait. En vert ce qui a été fait et n'était pas prévu}
  	\label{cal}
  \end{center}
\end{figure}

Finalement, le calendrier de départ a subi beaucoup de changements. D'une part, la partie de simulation des mouvements des drones a été abandonnée car le projet s'est focalisé sur la mise en place des métabots en premier. D'autre part, la partie communication ne s'est pas déroulée comme prévu. L'intégration de la communication avec i-score dans simulationRainOfMusic (intégration de l'API d'OSSIA dans simulationRainOfMusic) nous a pris du temps. Une fois l'API en place, la communication avec i-score se fait assez facilement. Ensuite, l'ajout des panels dans l'interface de simulationRainOfMusic et de la possibilité de modifier les paramètres des robots via ces panels a posé deux problèmes: la communication entre l'interface et le reste du logiciel, qui a donné lieu à la classe \verb|Parameter| et la synchronisation entre l'interface de simulationRainOfMusic et i-score. En effet, pour ce dernier, une boucle infinie peut facilement apparaître comme vu précédemment car les listener vers i-score et vers l'interface de simulationRainOfMusic déclenchent automatiquement la mise à jour des valeurs qu'ils écoutent. 

