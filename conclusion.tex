\section{Conclusion}

Dans cette partie, nous faisons le bilan de notre travail simulationRainOfMusic, de l'apport de notre logiciel au projet global et des améliorations envisageables.

\subsection{Bilan}

Nous avons implémenté un environnement 3D contenant des robots simulant une chorégraphie réalisée avec i-score. Pour la simulation, des paramètres comme la batterie et la perte de paquets ont été introduits. Concernant la communication, celle-ci est bi-directionnelle et cela permet à notre logiciel d'être aussi un logiciel d'écriture de chorégaphie. 

Bien que la partie drone n'ait pas été réalisée dûe à un manque d'informations, les besoins initiaux ont pu être satisfaits. De plus, la possibilité d'écrire avec notre logiciel nous paraît être une fonctionnalité très intéressante pour la suite du projet lors de l'écriture de la chorégraphie.

\subsection{Apport de notre travail dans le projet RainOfMusic}

Dans le cadre général du projet RainOfMusic, nous espérons que notre logiciel sera un support visuel mais aussi une aide pour les artistes à l'écriture de la chorégraphie. Ainsi, simulationRainOfMusic sera utilisé lors de la phase de la réalisation de la chorégraphie et de la simulation de cette dernière. 

L'ensemble de notre code étant en open source sur le dépôt GitHub, le logiciel pourra facilement être récupéré et amélioré pour correspondre au mieux aux besoins des artistes. 

Plus généralement, notre travail sur la communication bidirectionnelle avec i-score qui repose sur une classe existante d'openFrameworks (\verb|ofParameter|) pourra être utile dans un cadre plus large que ce projet. En effet, les utilisateurs d'i-score pourront synchroniser facilement avec openFrameworks.

\subsection{Améliorations envisageables}

Dans le cadre de travaux futurs sur le simulationRainOfMusic, plusieurs améliorations peuvent être envisagées.

Certains paramètres pourraient être définis dynamiquement via l'interface, nous pensons notamment aux dimensions de la scène qui pourraient être initialisées au lancement du logiciel. De la même façon, d'un point de vue ergonomique la possibilité de rajouter automatiquement des robots depuis i-score et dynamiquement depuis notre interface semble indispensable. Actuellement, ces deux paramètres sont définis dans le code, sachant que la chorégraphie est composé d'un nombre fixé de robots, c'est une amélioration qui n'est peut être pas prioritaire mais néanmois intéressante.

Une amélioration plus prioritaire du point de vue de la chorégraphie est celle qui a été mentionnée par les artistes et n'a pas pu être pris en compte dans la simulation. Il s'agit du fait que les metabots puissent changer leur hauteur. Il serait intéressant d'avoir un modèle 3D avec la possibilité d'articuler ses membres pour pouvoir représenter visuellement le changement d'hauteir. Cette problématique a été étudiée dans l'article visant à modéliser les différents mouvements des robots \cite{robotArt} présenté dans l'état de l'art. Nous pourrions donc envisager que le travail à fournir se baserait sur cette article, sachant que toutes les parties composant le metabot sont disponibles en modèles 3D.

Enfin, lorsque le projet introduira des drones, une modélisation physique de leurs mouvements pourra être implémentée dans le logiciel, pour obtenir une simulation proche de la réalité.

%\subsection{Apport personnel}
%peut être écrire un paragraphe chacun avec ce que ça nous apportait ..?