\section{Conclusion}

Dans cette partie, nous faisons le bilan de notre travail sur simulationRainOfMusic, de ce que le logiciel a apporté au projet global et des améliorations envisageables.

\subsection{Bilan}

tableau ou liste de check pour les fonctionnalités implémentées

\subsection{Apport de notre travail dans le projet RainOfMusic}

ce que notre travail a finalement apporté au projet dans globalité (possibilité de simuler mais aussi d'écrie la chorégraphie avec notre logiciel : un plus qui n'avait pas vraiment explicité dans les besoins au début)

\subsection{Améliorations envisageables}

Dans le cadre de travaux futurs sur le simulationRainOfMusic, plusieurs améliorations peuvent être envisagées.

On pourait par exemple initialiser la scène au lancement de simulatiuonRainOfMusic alors que cela se fait dans le code actuellement.

Dans le même esprit, les robots sont actuellement ajoutés depuis le code. Il serait préférable de pouvoir les ajouter automatiquement soit depuis i-score, soit depuis l'interface de simulationRainOfMusic.  

De plus, un aspect important de la chorégraphie a été mentionné par les artistes et n'est pas pris en compte dans la simulation. Il s'agit du fait que les metabots peuvent changer leur hauteur.

Enfin, quand il y aura plus d'informations sur les drones, il faudra implémenter leurs mouvements. 


%\subsection{Apport personnel}
%peut être écrire un paragraphe chacun avec ce que ça nous apportait ..?