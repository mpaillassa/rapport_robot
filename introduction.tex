\section{Introduction}

Le but du projet Rain of music est de pouvoir scénariser un spectacle impliquant des robots et de la musique. Deux types de robots sont envisagés: des métabots (robots terrestres) et des drones (robots volant sans pilote). Ces robots doivent réaliser une chorégraphie dans un espace donné et embarquer des haut parleurs pour émettre des sons. Ce projet pluridisplinaire implique trois parties: des étudiants en arts pour la conception et l'écriture de la chorégraphie, des étudiants roboticiens pour gérer les problématiques de communication et de localisation des robots, et des étudiants en technologies multimédia (nous-mêmes) pour les sons et l'interface qui permettra aux artistes d'écrire et de simuler des chorégraphies. \\

Notre travail s'inscrit dans le début du projet (3 premiers mois) qui durera bien plus longtemps pour arriver aux objectifs finaux cités précédemment. Dans ce contexte, nous avons dû adapter nos objectifs. Ainsi, nous nous sommes fixés de réaliser un logiciel qui permettra de simuler une chorégraphie dans un espace 3D pour aider les artistes dans l'écriture d'une chorégraphie. Nous allons d'abord préciser un peu les objectifs et les deux principales problématiques liées au logiciel.Comme dit précédemment, notre travail consistera à écrire un logiciel permettant de visualiser et de simuler une chorégraphie écrite par les artistes. Cette chorégraphie est écrite grâce au logiciel i-score, qui est un séquenceur: il permet d'écrire des scénarios en programmant des évènements dans le temps. Notre travail impliquera alors deux problématiques: 
\begin{itemize}
\item la récupération des données d'i-score dans le logiciel, de manière à connaître les positions des robots.
\item l'affichage des robots dans une scène 3D, qui impliquera d'autres problématiques comme la gestion de collisions.
\end{itemize} 

Dans ce rapport, nous allons d'abord faire un état de l'art. Ensuite, nous allons présenter les outils que l'on a choisi pour remplir nos objectifs. Puis, nous nous intéresserons à l'architecture du logiciel et son implémentation. Enfin, nous évaluerons le logiciel en testant plusieurs scénarios différents. \\