\section{Architecture et implémentation du logiciel}

\subsection{Architecture}

\subsection{Implémentation}

\subsubsection{Visualisation}

mise en place de la scène 3D très facilement avec une classe camera déjà toute prête dans openFrameworks (\verb|EasyCam|)

possibilité de rajouter des axes x y z en plus pour mieux situer le robot dans l'espace

avec la souris, possibilité de se déplacer et de zoomer dans la scène 3D 

importation de modèle 3D pour représenter les metabots

sélection des robots, affichage d'informations relatives au robot sélectionné (identifiant et type du robot, position et fréquence pour les metabots)

affichage de messages warning pour les robots qui rentrent en collision et la zone de colliison est indiquée à l'aide d'un cercle rouge dans la scène

ajout d'une interface pour visualiser/modifier les positions et les vitesses via des sliders, visualiser s'il y a des collisions ou non, lancer la chorégraphie (boutton play)

\subsubsection{Communication}

publication d'un arbre contenant le noeud 'scène', qui lui-même contient les noeuds pour les robots (chaque robot représente un sous-arbre contenant les différents attributs)

chaque attribut des robots est publié : la classe \verb|Parameter| qui permet cette publication encapsule la classe \verb|ofParameter| de openFrameworks. La classe \verb|ofParameter| permet de lier un listener à une valeur. Le listener permet de lier la valeur contenue dans \verb|ofParameter| à une autre fonction qui permet de mettre à jour cette valeur. Par exemple, un ofParameter peut être utilisée pour lier une valeur position en x avec un slider dans l'interface utilisateur : le fait de bouger le slider mettre à jour la valeur représentant la position en x. De cette façon, utiliser une variable de type \verb|ofParameter| permet de lier un attribut de robot à l'interface utilisateur. 
Il manque donc que le lien avec i-score : lorsqu'une valeur, dans l'arbre publié en réseau et affichée dans i-score, est modifiée via l'interface utilisateur d'i-score, il faudrait qu'au niveau de notre logiciel la valeur soit mise à jour. Sachant, que dans i-score lorsqu'une valeur de l'arbre est modifiée, l'arbre en réseau est directement mise à jour à l'aide du protocole Minuit.

Ainsi, nous avons choisi de réaliser une classe \verb|Parameter| qui a un listener (appelé \verb|Callback|) aussi vers i-score, lui permettant de mettre à jour sa valeur en accord avec i-score. Ainsi, la classe \verb|Parameter| encapsulant \verb|ofParameter|, a la possibilité de mettre à jour sa valeur à l'aide de ses deux listeners : l'un vers l'interface utilisateur de notre simulation, l'autre vers i-score.


\subsubsection{Simulation}

détection de collision manuelle : bibliothèques graphiques ne compilaient pas lors de l'implémentation de la collision (Bullet)...

