\section{Réalisation du logiciel}

Maintenant que nous avons vu les outils utilisés pour réaliser simulationRainOfMusic, nous allons présenter son architecture et les éléments importants de son implémentation.

\subsection{Architecture}
Dans les premières étapes de la réalisation du logiciel, nous avons établi l'architecture en respectant certains points qui nous paraissaient essentiels à ntore projet.

Tout d'abord, notre logiciel étant un logiciel de visualisation avant tout, il nous semblait plus clair de choisir une architecture modèle-vue. Cela nous permettait d'avoir une séparation nette entre la partie données des robots et la partie modélisation et affichage 3D. 

Ensuite, sachant que le projet contient des robots Metabot mais aussi des drones, il nous avait été demandé à ce que notre logiciel soit modulaire, facilitant ainsi l'ajout de nouvaux types de robots. C'est pour cela que nous avons choisi dans notre implémentation d'utiliser le mécanisme en C++ des \verb|template| pour assurer ce côté modulaire.

Enfin, une attention particulière a été portée à la classe \verb|Parameter|, jouant le rôle d'intermédiaire entre i-score et l'interface utilisateur de notre logiciel. Cette classe occupe une place importante dans le projet puisqu'elle permet de mettre à jour les valeurs en fonction des modifications apportées à notre interface et à i-score.

Par ailleurs, l'intégration de la classe \verb|Parameter| dans l'architecture s'est déroulée naturellement : elle prend le rôle du contrôleur qui met à jour les valeurs du modèle en fonction des données modifiées sur la vue qui est aussi l'interface utilisateur.

		
\subsection{Implémentation}

Dans cette partie, nous allons présenter simulationRainOfMusic en s'intéressant d'un peu plus près au code qui permettent de remplir les trois objectifs principaux: la visualisation, la communication et la simulation.

\subsubsection{Visualisation}

La mise en place de la scène 3D a été réalisée assez facilement car la caméra est déjà implémentée dans openFrameworks avec la classe (\verb|EasyCam|). L'utilisateur a la possibilité de se déplacer et de zoomer dans la scène 3D grâce à la souris. Les robots sont représentés par des modèles 3D.

%IMAGE screenshot logiciel (attention reference dans le paragraphe suivant)

L'utilisateur peut également sélectionner un robot. Lorsq'un robot est sélectionné, différentes caractéristiques le concernant s'affichent en haut à gauche de la fenêtre de simulation (reference a l'image). Ces caractéristiques sont son identifiant, son type (métabot ou drone) et sa position. La sélection se fait en détectant la couleur du robot.

IMAGE message de sélection


Si deux robots risquent de rentrer en collision, ceux-ci sont entourés d'un cercle rouge afin que ce soit bien visible et un message informatif indiquant la position et le robot entré en collision est affiché.

IMAGE message de warning et le cercle rouge

En plus de ces informations, un panel permet de visualiser et de modifier les positions et vitesses des robots via des sliders. Un emplacement dans l'onglet affiche également si les robots sont en collision ou pas. Une fois des vitesses choisies, un bouton play permet de voir la chorégraphie.

IMAGE screen shot GUI 

possibilité de rajouter des axes x y z en plus pour mieux situer le robot dans l'espace

échelle du repère : graduation au mètre
					
IMAGE image avec graduation

\subsubsection{Communication}

La communication avec i-score est possible grâce à l'API d'OSSIA et à la classe \verb|Network|. Cette dernière permet d'initier la communication avec i-score. A la construction d'un objet Network, on lance l'exécution d'une fonction sur un thread qui fait la publication: elle déclare le programme comme un objet réseau, que l'on peut ajouter dans i-score en lui donnant son nom et le numéro de port associé, et qui constitue la racine de l'arbre du réseau. Elle envoie ensuite sur le réseau le noeud 'scène' en le déclarant comme noeud fils du programme.

A l'instanciation d'un robot, celui-ci est ajouté à l'arbre du réseau comme fils du noeud 'scène'. Chacun de ses attributs est aussi publié. Lorsqu'un attribut dans l'arbre publié sur le réseau est modifié via l'interface utilisateur d'i-score, sa valeur est mise à jour dans simulationRainOfMusic grâce à une fonction (\verb|Callback|).

IMAGE i-score arbre

Pour ajouter la communication entre simulationRainOfMusic et son interface, la classe \verb|Parameter| a été créee. Elle encapsule la classe \verb|ofParameter| de openFrameworks. La classe \verb|ofParameter| permet de lier un listener vers l'interface à une valeur. Le listener est une fonctionnalité qui permet d'être constamment à l'écoute de toute modification opérée au niveau de l'interface. Par exemple, un ofParameter peut être utilisé pour lier une valeur position en x avec un slider dans l'interface utilisateur : le fait de bouger le slider va mettre à jour la valeur représentant la position en x. De cette façon, utiliser une variable de type \verb|ofParameter| permet de lier un attribut de robot à l'interface utilisateur. 

Ainsi, la classe \verb|Parameter| encapsulant \verb|ofParameter| utilise un listener \verb|Callback| vers i-score et un listener vers l'interface de simulationRainOfMusic.

images (*2) bidirectionnel

\subsubsection{Simulation}

Concernant la simulation, un des aspects le plus important et étroitement relié à la visualisation est l'échelle spatio-temporelle des mouvements des Metabots. En effet, d'après la documentation technique de ces robots \footnote{\url{https://github.com/Rhoban/Metabot/blob/master/docs/}}, les commandes de mouvements envoyées au robot sont en millimètre par seconde. Ainsi, dans notre logiciel, les vitesses envoyées à travers l'interface utilisateur sont en millimètre par seconde. Cela permet aux utilisateurs d'avoir une simulation de la chorégraphie proche de la réalité.

Ensuite, nous avons implémenté trois évènements qui peuvent arriver au cours de la chorégraphie et qui nous semblaient utiles à mettre en place pour l'artiste simulant la chorégraphie. 

Le premier est la panne du robot dûe à un niveau de batterie nul. Pour cela, un modèle simple linéaire en fonction de la distance parcourue a été implémenté. Néanmoins, ce modèle pourrait être modifié pour obtenir une modélisation plus fidèle à la consommation des batteries des metabots. Le niveau de consommation dse différents metabots est visible dans le panel simulation de l'interface graphique. 

Le deuxième évènement est la perte de paquets lors de la communication entre les robots et le logiciel envoyant les commandes i-score. Identiquement, au point précédent, un modèle simple a été implémenté dû fait que cette communication n'est pas encore établié dans l'état actuel du projet. Une probabilité de perte de paquets a été introduite dans notre implémentation, et permet de simuler les pertes de paquets lorsque notre simulation reçoit de i-score les modifications apportées aux données de l'arbre partagé. 

Lorsqu'un paquet est perdu, le metabot ne verra donc pas ses paramètres changées et gardera les précédentes. Concrètement, lors d'un chorégraphie cela signifiera que les metabots pourraient garder leur vitesse précédente et ne s'arrêterait pas. Ce comportement est fidèle à la réalité, puisque les metabots gardent leurs paramètres tant qu'ils ne reçoivent pas de commandes les modifiant. 

Une fois que le protocole de communication entre les robots et i-score sera en place, des tests pourraient être faits et permettront d'obtenir une idée approximative du pourcentage de perte de paquets, celle-ci pourra être reflétée dans notre logiciel de simulation.

IMAGE panel de simulation

Le troisième évènement qui serait peut être le plus fréquent lors d'une simulation réelle de la chorégraphie avec les robots est la possibilité de collision entre les différents robots. Pour cela, une première approche "manuelle" a été implémentée qui se basaient sur un système simple vérifiant la taille des robots et leur position. Par la suite, la possibilité d'utiliser des bibliothèque de moteur physique a été étudiée mais n'a pas aboutie. La raison principale à cela a été que ces bibliothèques (ofxBullet, ofxMSAPhysics) ne pouvaient être compilées suite à des erreurs dans le code publié sur GitHub. Il est fort probable que ce problème était dû à des commits récents mais nous n'avons pas insisté dans la suite du projet pour recompiler ces bibliothèques. 

